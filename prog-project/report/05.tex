\documentclass[main]{subfiles} 
\graphicspath{{img/}}


\begin{document}

\section{Results}
The implementation of all the programs contained in the project result in obtaining the data to build the graphical user interface \ac{gui} 
for the user to get information on the lausanne housing market and navigate easily between available properties.

\subsection{Dataset}
The dataset we used for the building the \ac{gui} is reduced compared to all the data we initially have access to. 
We chose to keep this reduced version of the dataset, to keep the \ac{gui} easy to use and read for the user. 
The following image shows the the first few rows of the dataset we used.


\subsection{\ac{gui}}
The \ac{gui} presents the scrapped data in an orderly manner.
\subsubsection{Main Window}
The main window contains the four available tabs namely, price range, Rooms, Zip Code and Graphs and the empty treeview as shown below.

\subsubsection{Frame 1}
The first frame allows the user to search within a specific price range. 
The user enters a minimum value and maximum value within each of the boxes and presses the search button. 
The treeview return all the properties within chosen price range. 
To restart the process, the user can either press the clear button or enter new values.

\subsubsection{Frame 2}
The second frame allows the user to search within a specific number of rooms. 
The user enters a minimum value and maximum value within each of the boxes and presses the search button. 
The treeview return all the properties within the chosen number of rooms. 
To restart the process, the user can either press the clear button or enter new values.

\subsubsection{Frame 3}
The third frame allows the user to search for properties in a specific zip code. 
The user enters the wished zip code in the box and presses the search button. 
The treeview return all the properties within the chosen zip code. 
To restart the process, the user can either press the clear button or enter a new value.

\subsubsection{Frame 4}
The fourth frame displays two buttons corresponding to two different bar graphs, 
namely the average price by zip code and the average price by rooms.
The user must simply press on the corresponding button and the graph is displayed. 
figure nb displays the average price per zip code. At the time when we run the program we can see that the zip code correspnding to x
has a much higher price. Why?+PICTURES

figure nb displays the average price per rooms. At the time when we run the program we can see that the zip code correspnding to x
has a much higher price. Why?

\subsubsection{Heatmap}
Iniatially we wanted to display a heatmap of Lausanne based on the average price by zip code. 
This map is built with the help of the geopandas and folium packages. 
To build it we merge the MeanPriceZip file with the geolocations of the Lausanne's zip codes. Nonetheless, 
due to a merging and geolocation issue we were unable to display the map. 
\end{document}