\documentclass[main]{subfiles} 
\graphicspath{{img/}}


\begin{document}

\begin{abstract}
    Real estate has been for a long time an opaque, decentralized market. As such, the Swiss housing market is subject to information asymmetry between buyers and sellers, creating potential adverse selection. The housing market is dominated by real estate agents who possess more information than potential buyers which contributes to skewing the price of houses.
    Though transparency has positively evolved with the apparition of the internet and online properties platforms, there is still a lack of publicly available historical data and comparative databases.
    As such, our research is focused on testing whether creating a continuously updated open-source database of all present and historical listings in Switzerland is feasible. 
    
\end{abstract}
    
\begin{IEEEkeywords}
swiss real estate, web scraping, comparis, scrapy, selenium, user interface, transparency, economic research
\end{IEEEkeywords}
    
\section{Introduction}
Real Estate Transparency has long been associated with favorable business environments \cite{RealEstateTransparency}.
 While Switzerland is ranked 11th in the JLL “Global Real Estate Transparency Index” \cite{GlobalRealEstate}, 
 it is lagging many other European countries, such as France, the UK, Sweden, and Germany. 
By empirical observation, the Swiss market is dominated by private valuers and brokers (“Agents”), 
who refrain from citing exact addresses and informing buyers the right price of properties on the market. 
Hence, leading to vast information asymmetry between Agents and potential buyers.
However, the rise of web platforms is slowly erasing information asymmetry,
 where potential buyers may compare different properties without the use of Agents. 
 Nevertheless, even with the use of these revolutionary real estate web platforms, 
 it is still difficult to assess for a potential buyer what drives the price of their target property. 
 Is it undervalued? Overvalued? At the right price?
Therefore, despite advancements in enhancing transparency of the housing market, 
there are still improvements to be made to minimize information asymmetry, 
so that potential buyers may make rational and informed decisions before investing in real property.



\end{document}