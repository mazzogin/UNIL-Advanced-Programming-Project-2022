\documentclass[main]{subfiles} 
\graphicspath{{img/}}


\begin{document}

\begin{abstract}
Last thing that will be written. Lorem ipsum dolor sit amet, consectetur adipiscing elit, sed do eiusmod tempor incididunt ut labore et dolore magna aliqua. Mauris cursus mattis molestie a iaculis at erat pellentesque adipiscing. Vitae auctor eu augue ut lectus arcu bibendum. Dui id ornare arcu odio ut sem. Tellus at urna condimentum mattis. Sed euismod nisi porta lorem mollis aliquam. Orci eu lobortis elementum nibh tellus molestie. Posuere ac ut consequat semper viverra nam libero. 
\end{abstract}
    
\begin{IEEEkeywords}
swiss real estate, web scraping, comparis, scrapy, selenium, user interface, transparency, economic research
\end{IEEEkeywords}
    
\section{Introduction}
Transparency in the real estate market has long been associated with favorable business environments. 
While Switzerland is ranked 11th in the JLL “Global Real Estate Transparency Index”, 
there are only a few platforms that may allow end-market investors to compare prices and assess the worth of a property.
The Swiss market is dominated by \hly[private valuers] and brokers (“Agents”),
who refrain from citing exact addresses and informing buyers the right price of properties on the market. 
Hence, leading to vast information asymmetry between Agents and potential buyers.
However, the rise of web platforms is slowly erasing information asymmetry, 
where potential buyers may compare different properties without the use of Agents. 
Nevertheless, even with the use of these revolutionary real estate web platforms, 
it is still difficult to assess for a potential buyer what drives the price of their target property. 
Is it undervalued? Overvalued? At the right price?
Therefore, despite advancements in enhancing transparency of the housing market, 
there are still improvements to be made to minimize information asymmetry, 
so that potential buyers may make rational and informed decisions before investing in real property.


\end{document}