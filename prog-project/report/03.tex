\documentclass[main]{subfiles} 
\graphicspath{{img/}}


\begin{document}

\section{Implementation}
\begin{itemize}
    \item \pkg[Scrapy] -  version $2.6.1$
    \item \pkg[Selenium] - version $4.1.5$
    \item \pkg[Webdriver\_manager] - version $3.5.4$
    \item \pkg[Numpy] -  version $1.22.3$
    \item \pkg[Pandas]  - version $1.4.2$
    \item \pkg[Time]
    \item \pkg[Datetime]
    \item \pkg[Openpyxl] - version $3.0.9$
    \item \pkg[Tk (tkinter)] - version $0.1.0$
    \item \pkg[Pillow] - version $9.1.1$
\end{itemize}

\subsection{Structure of the project}

\subsection{Implementation of the webscrapping}

\subsection{Implementation of the \ac{gui}}

\subsubsection{Organizing the scrapped data}
The first step to building the \ac{gui} is organizing the data with which we will be working.
We first analyze the dataset and then clean it for the \ac{gui} to function correctly.
This is done at the beginning of the program main.py.

In this program, we first load the dataset. 
Then we print the headings and the description to check the data and the types.
We proceed to split the address from the zip code, to get the zip code in a separate column. 
We do this as we want to be able to search properties within a certain zip code. 
We then remove variables that are misspelled or incorrect. 
Finally, we save the database. This database is the one that the \ac{gui} will use to return information to the user.

\subsubsection{Building the \ac{gui}}

\subsection{Maps?}

\end{document}