\documentclass[main]{subfiles} 
\graphicspath{{img/}}


\begin{document}

\section{Research Question}
\subsection{Problem}
Real Estate has long been a very opaque asset class, with subjective pricing methods; 
as opposed to asset classes that trade in the open market (stocks, bonds, derivatives, etc.). 
Historical transaction data is difficult to obtain, and the fact that each property is radically 
different in its characteristics makes it difficult for inexperienced buyers to assess property value. 

Therefore, buyers must rely on Agents, who have better knowledge of the housing market. 
\hly[Rutherford et al. (2005)] show that Agents sell their own houses for a premium of approximately $4.5\%$, 
whereas \hly[Levitt and Syverson (2008)] find that houses owned by Agents sell for approximately $3.7\%$ more 
than other houses. The evidence suggests the presence of Agents' information advantage in the housing market. 
Why do Agents pay lower prices when buying their own houses? 
One explanation is related to information asymmetries in the housing market. 
Real estate agents have information advantages over less informed “nonagent” buyers. 
These studies also show that Agents will use this information to their own personal advantage, 
which may enter in conflict with the interest of their clients.

With the apparition of e-commerce real estate platforms, potential buyers have access to current listings with prices, characteristics,
and pictures. This may help in reducing information asymmetry to a certain extent. 
However, these listing are actual listings, where historical listings are not available. 
Moreover, the data is not formatted and structured into databases to make informed comparisons. 
Finally, the price that is reflected in listings may not be the final prices, 
as there is generally a negotiation process involved before the transaction closes.

Therefore, we find the lack of a real open-source platform, showing past listings in a structured fashion 
to be necessary in reducing information asymmetry and enhancing transparency of the housing market in Switzerland.

\subsection{Objective}
Our goal is to provide Swiss potential home buyers with an open-source platform, 
that lists all previous past listings, which are ordered by NPA (Swiss postal codes), 
price, square meters and number of rooms.
This platform would pull data automatically in certain periodic intervals, 
which would provide with a wide range of historical data, useful to understand trends. 
This platform would give an edge both to professional Agents and potential buyers,
effectively reducing information asymmetry between the two parties.

Such platforms would also be relevant for statistics by authorities, insurances, mortgage lenders, 
as well as advertisers.

\subsection{Scope}
Considering the wide array of properties in Switzerland, 
we decided to focus our attention to Lausanne and its suburbs. 
This will permit us to test out the first version of our platform, 
with limited property data and historical data. 
However, the platform is scalable to become a nationwide platform. 

\end{document}